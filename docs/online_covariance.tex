\documentclass{article}

\usepackage{amsmath,amssymb,amsthm}

\newtheorem{definition}{Definition}
\newtheorem{result}{Result}
\newcommand{\set}[1]{\{#1\}}

\newcommand{\defref}[1]{Def~\ref{#1}}


\newcommand{\hapP}[1]{x_{#1}^{(p)}}
\newcommand{\hap}[1]{x_{#1}}
\newcommand{\bhapP}[1]{\bar{x}_{#1}^{(p)}}


\newcommand{\deltaP}[1]{\delta_{#1}^{(p)}}


\newcommand{\hcovarianceM}[1]{\text{cov}^{(#1)}}
\newcommand{\hcovM}[1]{C^{(#1)}}

\author{Robert Vogel}
\title{A recursion relaion for online coviarnce
of haplotype counts}
\date{2025-01-15}

\begin{document}

\maketitle


In this short write-up I document a recursion relation for single-pass,
or online, estimation of the unbiased covariance of haplotype counts.  This is
necessary because our genetic data sets are just too large to load into memory 
and for inefficient algoritms.  Indeed,
such a relationship is well established for the covariance, \cite{wikiOnlineCov},
however I was unsure how this procedure would extend to my definition of
haplotype count covariance.  In what follows I define the haplotype count
covariance, hcov for short, and then show the recursion relation that I'll
implement in C++.

I begin by introducing the quantities of interest.  Denote the expected number of
haplotype $p$ copies, at locus $m$, and for sample $i$ as $\hapP{mi} \in [0,2]$.
Importantly, the $\hapP{mi}$ is a real number as it represents an \emph{expected}
copy number, and not the actual copy number.  The number of distinct haplotypes are
the number of $K$ founder strains in our heterogenous stock rat colony making 
$p\in\set{1,2,\dots,K}$.  A locus $m\in\set{1,2,\dots,M}$ is any integer from 1 to
the $M$ total number of loci.  Lastly, we assume that the population consists of 
$N$ animals, i.e. samples.

As we are interested in the single-pass computation of the covariance, we must
first define the haplotype mean and covariance statistics.

\begin{definition}\label{def:hmean}
    The mean of expected haplotype counts for haplotype $p$ is
    %
    \begin{equation}\nonumber
        \bhapP{Mi} = \frac{1}{M} \sum_{m=1}^M \hapP{mi}.
    \end{equation}
    %
\end{definition}

Next the unbiased covariance over markers and haplotypes.

\begin{definition}\label{def:hcov}
    The unbiased covariance of expected haplotype counts over $p$ distinct haplotypes
    and $M$ markers is defined as
    %
    \begin{equation}\nonumber
        \hcovarianceM{M}(\hap{i},\hap{j}) = \frac{1}{M-1}\sum_{p=1}^K\sum_{m=1}^M
            \left(\hapP{mi} - \bhapP{Mi}\right)
            \left(\hapP{mj} - \bhapP{Mj}\right)
    \end{equation}
    %
    for which we will use $\hcovM{M}_{ij}$ as shorthand.
\end{definition}

The aim for this analysis is to write the mean \defref{def:hmean}
and covariance \defref{def:hcov} as a recursion relation.  That is, 
suppose there are a total of $M'$ loci in which we want an estimate
of the mean and covariance.  The statistics will be computed by a single
\texttt{for} loop over the total number of markers $M'$.  Meaning that 
for any iteration $1\leq M\leq M'$ we must figure out how to update the mean
and covariance estimates from the previous $M-1$ observations.  That is
we recursively update our estimates of the mean and covariance.  When we have
updated our mean and covariance estimates for all $M'$ markers, the \texttt{for}
loop will terminate and we'll have our desired estimates all without 
loading the entire data set at one time.

In presenting the recursion, let us define the quantity $\deltaP{Mi}$
%
\begin{definition}\label{def:delta}
    The value $\deltaP{Mi}$ is a quantity for updating the mean and covariances
    from iteration $M-1$ to iteration $M$ of sample $i$ and is defined as follows
    %
    \begin{equation}\nonumber
        \deltaP{Mi} = \hapP{Mi} - \bhapP{(M-1)i}.
    \end{equation}
\end{definition}

\begin{result}\label{res:mean}
    \begin{equation}\nonumber
        \bhapP{Mi} = \bhapP{(M-1)i} + \frac{\deltaP{Mi}}{M}
    \end{equation}
\end{result}

\begin{result}\label{res:cov}
    \begin{equation}\nonumber
        \hcovM{M}_{ij} = \frac{M-2}{M-1}\hcovM{M-1}_{ij} 
            + \frac{1}{M}\sum_{p=1}^K\deltaP{Mi}\deltaP{Mj}
    \end{equation}
\end{result}

\bibliography{refs}
\bibliographystyle{plain}

\end{document}
